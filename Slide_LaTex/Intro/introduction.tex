% -*- LaTeX -*-

% Add proof example to 54
% Resolution: inference rule, proof trees, semantic interpretation,
% rule interpretation, soundness and completeness

\documentclass[handout,x11names,unknownkeysallowed]{beamer}
%\documentclass[x11names]{beamer}
\usepackage{beamerthemeUAB}
\usepackage{verbatim}
\usepackage{color}
\usepackage{multirow}
\usepackage{cite}
%\usepackage{amsthm}
%\setbeamertemplate{theorems}[numbered]
%\newtheorem{proposition}{Proposition}
%\newtheorem{cor}{Corollary}
% -*- LaTeX -*-

\usepackage{subfigure,bm}
\usepackage{multicol}
\usepackage{amsmath}
\usepackage{epsfig}
\usepackage{graphicx}
\usepackage[all,knot]{xy}
\usepackage{marvosym}
\xyoption{arc}
\usepackage{url}
\usepackage{multimedia}
\usepackage{hyperref}
\usepackage[english]{babel}
\usepackage[latin1]{inputenc}
\usepackage{times}
\usepackage{caption}
%\usepackage[T1]{fontenc}
% Or whatever. Note that the encoding and the font should match. If T1
% does not look nice, try deleting the line with the fontenc.
\newcommand{\conv}{\stackrel{\mathbb{D}}{\longrightarrow}}
\newcommand{\cons}{\stackrel{p}{\longrightarrow}}
\newcommand{\nc}{\newpage \clearpage}
\newcommand{\etal}{\textit{et al.}}

\def\boldp{\mathbf p}
\def\btheta{\bm \theta}
\def\bpi{\bm \pi}

\title[] % (Called "`Short Title"', optional, use only with long paper titles)
{Introduction}

%\subtitle{EPID 753} % (optional)

\author[Dustin Long, PhD] % (optional, use only with lots of authors)
{Dustin~Long, PhD}


\institute[UAB]
{
  Department of Biostatistics\\
	University of Alabama at Birmingham

}

\def\insertcopyright{$\copyright$ 2019 by Dustin Long}
\def\insertslideinfo{\insertshorttitle}

\subject{Introduction}
% This is only inserted into the PDF information catalog. Can be left
% out. 



% This code is not needed since the logo is on every page in the lower left-hand corner
%\pgfdeclareimage[height=0.5cm]{university-logo}{unc-gillings-school-of-public-health-logo.png}
%\logo{\pgfuseimage{university-logo}}


% Delete this, if you do not want the table of contents to pop up at
% the beginning of each subsection:
%\AtBeginSubsection[]
%{
%   \begin{frame}<beamer>
%     \frametitle{Outline}
%     \tableofcontents[currentsection,currentsubsection]
%   \end{frame}
% }


% If you wish to uncover everything in a step-wise fashion, uncomment
% the following command: 

%\beamerdefaultoverlayspecification{<+->}

%\input macros.tex

\date[Introduction]{August 27, 2019}

\newcommand{\beamitem}{\begin{itemize}[<+-|alert@+>]}
%\newcommand{\beamitem}{\begin{itemize}}
%\newcommand{\etal}{\textit{et al.}}

%%%%%%%%%%%%%%%%%%%%%%%%%%%%%%%%%%%%%%%%%%%%%%%%%%%%%%%%%%%%%%%%%%%%%%
\begin{document}

\begin{frame}
  \titlepage
\end{frame}

\begin{frame}
Outline:
\begin{itemize}
\item Syllabus
\item Lecture
\item Homework
\item SAS and R
\item Review of BIOS 621/622
\end{itemize}

\end{frame}


%%%%%%%%%%%%%%%%%%%%%%%%%%%%%%%%%%%%%%%%%%%%%%%%%%%%%%%%%%%%%%%%%%%%%%
\section{Syllabus}
\begin{frame}
\beamitem
\item Please refer to syllabus as the contract between you and the instructor
\end{itemize}
\end{frame}

\section{Lecture}
\begin{frame}
\beamitem
\item Slides will be provided but may be considered minimal
\item The majority of most lectures will be on the board 
\end{itemize}
\end{frame}


\section{Homework}
\begin{frame}
\beamitem
\item All homework will be typed
\item SAS log file or R code will be submitted as a \textbf{separate file}.  I do not want to see undigested SAS output or any code in the submitted HW file.  The only exception to this is Markdown.  If you know what that is, you know why. 
\item Homework will be posted on CANVAS site and should be submitted there (unless I have the deadline incorrect, then email me).
\end{itemize}
\end{frame}

\section{Homework}
\begin{frame}
\beamitem
\item I recommend using \LaTeX\ .
\item Use online compilers, such as Overleaf or ShareLaTex
\item I use MikTex and TeXnicCenter and can assist with those installations
\item Markdown is also encouraged
\end{itemize}
\end{frame}

\section{SAS and R}
\begin{frame}
\beamitem
\item I will focus on SAS and R.  
\item Linear models are fairly easy in both SAS and R, I find SAS to be easier as options are all within one or two PROCS while R generally takes more time to set-up and get results
\item R has SUPERIOR graphics but generally not as useful at getting certain jobs out of school, such as FDA or pharma
\item However, it does make you more likely to get other jobs, such as data science
\item You should use both for exercises throughout this course even if you strongly favor one
\end{itemize}
\end{frame}

\begin{frame}
\beamitem
\item For R, I highly recommend using RStudio
\item Helps with typesetting and functioning
\item Utilize Google for help
\item Everyone should have taken or should take BST 680 with Dr. Jaeger
\end{itemize}
\end{frame}

\section{Review of BIOS 621/622}
\begin{frame}
\beamitem
\item Hypothesis testing
\item Five parts to a good hypothesis test for this or any class 
\item Null and alternative hypothesis, test statistic, p-value (or critical value), decision, and INTERPRETATION
\item All five are necessary to get full credit on homework and exams
\end{itemize}

\end{frame}


\begin{frame}
\beamitem
\item Review of topics you already know and love
\item One sample $z$ and $t$ tests
\item Two sample $t$ tests
\item Linear regression (but very little)
\end{itemize}
\end{frame}

\begin{frame}
What connects all those topics?
\beamitem
\item Continuous outcome
\item Normal assumption
\item ??
\end{itemize}
\end{frame}

\begin{frame}
This semester we will cover:
\beamitem
\item Simple (straight-line) Linear regression in depth
\item Multiple regression in two forms: epidemiological and prediction
\item ANOVA 
\item Matrix forms and Likelihood based
\end{itemize}
\end{frame}

\begin{frame}
\begin{center}
Questions?
\end{center}

\end{frame}



\end{document}
